\documentclass[11pt]{amsart}
%%% WARNING: Do NOT change the page size, fonts, or margins!  Penalties will apply.

\usepackage{graphicx, hyperref}
\usepackage{amssymb,amsmath,amsthm, mathtools}
\usepackage{placeins} %enables \FloatBarrier, that prevents floats from going below it.
\usepackage{caption}
\usepackage{subcaption}
\usepackage{algpseudocode, algorithm}
\usepackage{tikz}
\usepackage{physics}
\usepackage[T1]{fontenc}
\usepackage{DejaVuSansMono}
\usetikzlibrary{arrows}
\usetikzlibrary{tikzmark}
\usepackage{listings}

% Define style for Python code snippets
\lstset{
    language=Python,
    basicstyle=\small\ttfamily, % Set font size to small
    frame=single, % Add frame around code snippets
    showstringspaces=false % Don't show spaces in strings
}

%%% WARNING: Do NOT change the page size, fonts, or margins!  Penalties will apply.
%%% WARNING: Do NOT change the page size, fonts, or margins!  Penalties will apply.

% Some macros for ease of use
\newcommand{\R}{{\mathbb R}}
\newcommand{\A}{{\mathrm{A}}}


\begin{document}
\title{Helping the Spacing Guild Navigate to Arrakis}
\author{Cannon Tuttle, Curtis Evans, Spencer Ashton, Tyler Sanders}

%% comment out next command to put today's date after names of group members, or put a desired day in the parethesis
\date{\today}

\begin{abstract}
    
\end{abstract}

\maketitle

%% First Section
\section{Problem Statement and Motivation}
The Guild Navigators tasked with delivering House Atreides to Arrakis are unsure what the best path is to get from Caladan to Arrakis, but they know of a planet in another universe, Earth, where there 
are smart ACME students learning about optimal control. They want to be able to figure out with a given travel time what the optimal path and control is, and they want to know what the optimal final time 
would be to see if they have enough feasible fuel with that optimal control. Since the ACME students are from another universe and solar system they have decided to focus on getting from their home planet, 
Earth, to other planets in their own solar system. 

\section{State Equations}
We used Newton's second law of motion $F=ma$ to arrive at our state equations where $m_p$ is the mass of the planet, $\mathbf{x}_s$ is the $x$ and $y$ position of the space ship, and $\mathbf{x}_p$ is the $x$
and $y$ position of the planet, and $\mathbf{u}$ is the acceleration in the $x$ and $y$ direction for the spaceship. \textcolor{red}{TODO} talk about what $G$ is and how this equation makes sense.
\[\ddot{\mathbf{x}} = -G\sum_{p\in{P}}^{}\frac{m_p}{||\mathbf{x}_s-\mathbf{x}_p||_2^3}(\mathbf{x}_s-\mathbf{x}_p) + \mathbf{u}\]
Converting $\ddot{\mathbf{x}}$ into a first order differential equation we get the following state equation. 
\[\begin{pmatrix}
    \dot{x}_s \\
    \dot{y}_s \\
    \ddot{x}_s\\
    \ddot{y}_s 
\end{pmatrix} = \begin{pmatrix}
    \dot{x}_s \\
    \dot{y}_s \\
    -G\sum_{p\in{P}}^{}\frac{m_p(x_s - x_p)}{((x_s-x_p)^2+(y_s-y_p)^2)^{3/2}} + u_x \\
    -G\sum_{p\in{P}}^{}\frac{m_p(y_s - y_p)}{((x_s-x_p)^2+(y_s-y_p)^2)^{3/2}} + u_y
\end{pmatrix}\]

We decided to minimize the control used to get the optimal fuel usage. That gives us the following cost functional. 
\[J[u] = \int_{0}^{t_f}||\mathbf{u}||_2^2dt\]

We are now ready to use Pontryagin's maximum principle with 
\[H = \mathbf{p}\cdot\mathbf{f(\mathbf{x})} - L\]
\[= p_1\dot{x}_s + p_2\dot{y}_s + p_3(-G\sum_{p\in{P}}^{}\frac{m_p(x_s - x_p)}{((x_s-x_p)^2+(y_s-y_p)^2)^{3/2}} + u_x)\]
\[+ p_4(-G\sum_{p\in{P}}^{}\frac{m_p(y_s - y_p)}{((x_s-x_p)^2+(y_s-y_p)^2)^{3/2}} + u_y) - u_x^2 - u_y^2\]

This give us the following co-state evolution equations
\[\dot{p}_1 = -\frac{\partial H}{\partial x_s}\]







%%%%%%%%%%%%%%%%%%%%%%%%%%%%%%%%%%%%%
%% Bibliography below
%%%%%%%%%%%%%%%%%%%%%%%%%%%%%%%%%%%%%
\FloatBarrier % Keep the figures from being put after the bibliography
\newpage
%% If using bibtex, leave this uncommented
%\bibliography{refs} %if using bibtex, call your bibtex file refs.bib
\bibliographystyle{alpha}

%% If not using bibtex, comment out the previous two lines and uncomment those below
\begin{thebibliography}{99}
\bibitem{Newton} William M., Samuel L., Jeff S. University Physics Volume 1. 2021 OpenStax. https://openstax.org/details/books/university-physics-volume-1 

\end{thebibliography}

\end{document}