\documentclass[11pt]{amsart}
%%% WARNING: Do NOT change the page size, fonts, or margins!  Penalties will apply.

\usepackage{graphicx, hyperref}
\usepackage{amssymb,amsmath,amsthm, mathtools}
\usepackage{placeins} %enables \FloatBarrier, that prevents floats from going below it.
\usepackage{caption}
\usepackage{subcaption}
\usepackage{algpseudocode, algorithm}
\usepackage{tikz}
\usepackage{physics}
\usepackage[T1]{fontenc}
\usepackage{DejaVuSansMono}
\usetikzlibrary{arrows}
\usetikzlibrary{tikzmark}
\usepackage{listings}

% Define style for Python code snippets
\lstset{
    language=Python,
    basicstyle=\small\ttfamily, % Set font size to small
    frame=single, % Add frame around code snippets
    showstringspaces=false % Don't show spaces in strings
}

%%% WARNING: Do NOT change the page size, fonts, or margins!  Penalties will apply.
%%% WARNING: Do NOT change the page size, fonts, or margins!  Penalties will apply.

% Some macros for ease of use
\newcommand{\R}{{\mathbb R}}
\newcommand{\A}{{\mathrm{A}}}


\begin{document}
\title{Helping the Spacing Guild Navigate to Arrakis}
\author{Cannon Tuttle, Curtis Evans, Spencer Ashton, Tyler Sanders}

%% comment out next command to put today's date after names of group members, or put a desired day in the parethesis
\date{\today}

\begin{abstract}
    
\end{abstract}

\maketitle

%% First Section
\section{Problem Statement and Motivation}
The \textbf{Dune} Guild Naviators are tasked with delivering House Atreides from the planet Caladan to Arrakis. Usually they solve the optimal control in their mind while taking spice, but since spice production
has been disrupted they've instead decided to turn to ACME students on Earth to solve the optimal control for them. They have asked the ACME students to find the best path while minimizing fuel consumption.


\section{State Equations}
We used Newton's second law of motion $F=ma$ to arrive at our state equations where $m_p$ is the mass of the planet, $\mathbf{x}_s$ is the $x$ and $y$ position of the space ship, and $\mathbf{x}_p$ is the $x$
and $y$ position of the planet, and $\mathbf{u}$ is the acceleration in the $x$ and $y$ direction for the spaceship. \textcolor{red}{TODO} talk about what $G$ is and how this equation makes sense.
\[\ddot{\mathbf{x}} = -G\sum_{p\in{P}}^{}\frac{m_p}{||\mathbf{x}_s-\mathbf{x}_p||_2^3}(\mathbf{x}_s-\mathbf{x}_p) + \mathbf{u}\]
Converting $\ddot{\mathbf{x}}$ into a first order differential equation we get the following state equation. 
\[\begin{pmatrix}
    \dot{x}_s \\
    \dot{y}_s \\
    \ddot{x}_s\\
    \ddot{y}_s 
\end{pmatrix} = \begin{pmatrix}
    \dot{x}_s \\
    \dot{y}_s \\
    -G\sum_{p\in{P}}^{}\frac{m_p(x_s - x_p)}{((x_s-x_p)^2+(y_s-y_p)^2)^{3/2}} + u_x \\
    -G\sum_{p\in{P}}^{}\frac{m_p(y_s - y_p)}{((x_s-x_p)^2+(y_s-y_p)^2)^{3/2}} + u_y
\end{pmatrix}\]

We decided to minimize the control used to get the optimal fuel usage. That gives us the following cost functional and boundary conditions. 
\[J[u] = \int_{0}^{t_f}||\mathbf{u}||_2^2dt\]
\[\mathbf{x}_s(0) = \text{Caladan's Position},\: \dot{\mathbf{x}}_s(0) = \text{Caladan's Velocity}\]
\[\mathbf{x}_s(t_f) = \text{Arrakis' Position},\: \dot{\mathbf{x}}_s(t_f) = \text{Arrakis' Velocity}\]

We are now ready to use Pontryagin's maximum principle with 
\[H = \mathbf{p}\cdot\mathbf{f(\mathbf{x})} - L\]
Applying this to our state equation and Lagrangian we get the following Hamiltonian.
\[H = p_1\dot{x}_s + p_2\dot{y}_s + p_3(-G\sum_{p\in{P}}^{}\frac{m_p(x_s - x_p)}{((x_s-x_p)^2+(y_s-y_p)^2)^{3/2}} + u_x)\]
\[+ p_4(-G\sum_{p\in{P}}^{}\frac{m_p(y_s - y_p)}{((x_s-x_p)^2+(y_s-y_p)^2)^{3/2}} + u_y) - u_x^2 - u_y^2\]

This give us the following co-state evolution equations by using $\mathbf{p}' = \frac{DH}{D\mathbf{x}}$
\begin{align*}
    \dot{p}_1 &= p_3G[\sum_{p\in{P}}^{}\frac{m_p}{((x_s-x_p)^2+(y_s-y_p)^2)^{3/2}} - \frac{3m_p(x_s - x_p)^2}{((x_s-x_p)^2+(y_s-y_p)^2)^{5/2}}] \\
    \dot{p}_2 &= p_4G[\sum_{p\in{P}}^{}\frac{m_p}{((x_s-x_p)^2+(y_s-y_p)^2)^{3/2}} - \frac{3m_p(y_s - y_p)^2}{((x_s-x_p)^2+(y_s-y_p)^2)^{5/2}}] \\
    \dot{p}_3 &= -p_1 \\
    \dot{p}_4 &= -p_2. \\
    p_1&(t_f)= p_2(t_f) = 0 = p_3(tf) = p_4(t_f)
\end{align*}
Now we need to find $\tilde{\mathbf{u}}$ by maximizing the Hamiltonian which gives us the following
\[\tilde{u}_x = \frac{p_3}{2} \:\:\text{and}\:\: \tilde{u}_y = \frac{p_4}{2}.\]


We now have a system of differential equations that can be solved using \textcolor{red}{TODO} solve\_bvp.

\subsection{Planetary Motion}
Since the ACME students aren't familiar with the planetary motion of Arrakis and Caladan they decided to look at getting from Mars to Earth with their respective masses and orbits.
We have made a planet class that will give us the planets mass, position, and velocity.





%%%%%%%%%%%%%%%%%%%%%%%%%%%%%%%%%%%%%
%% Bibliography below
%%%%%%%%%%%%%%%%%%%%%%%%%%%%%%%%%%%%%
\FloatBarrier % Keep the figures from being put after the bibliography
\newpage
%% If using bibtex, leave this uncommented
%\bibliography{refs} %if using bibtex, call your bibtex file refs.bib
\bibliographystyle{alpha}

%% If not using bibtex, comment out the previous two lines and uncomment those below
\begin{thebibliography}{99}
\bibitem{Newton} William M., Samuel L., Jeff S. University Physics Volume 1. 2021 OpenStax. https://openstax.org/details/books/university-physics-volume-1 

\end{thebibliography}

\end{document}